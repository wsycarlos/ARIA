\documentclass[UTF8]{ctexart}
\title{项目进程}
\author{wsycarlos}
\date{\today}
\begin{document}
\maketitle
\section{第一阶段(2015.9.30-2015.12.18)}

\subsection{初始化阶段}

\subsubsection{国庆期间(2015.9.30-2015.10.07)}
\paragraph{主要任务}
\subparagraph{全体}
熟悉大航海家3

要求:

1)教学模式完全通过

2)新建存档达成旅行商人级别

3)当过海盗

4)办过庆典

5)结过婚

6)建造三个以上工厂形成产业链和贸易路线并运行五年以上

7)大致了解电子书中有哪些内容


提示:使用+/-按键可以快捷加快减慢游戏速度,神器。先运行免显卡测试补丁或者P3Setup 然后出错使用兼容性模式选择XP运行

\subparagraph{数据库端}
选定数据库,搭建环境,和服务器完成连接读取调试,建立用户数据表,初步完成玩家登录操作。

\subparagraph{客户端}
选择网络类型(http or socket),选定网络层负责人,选择UI框架,根据大航海家3大体模型,找到各界面所需素材类型,开始填坑(登录界面,大地图界面,城镇界面,城镇下各地方界面(如酒馆))。着手准备开发登录界面。

\subparagraph{其余}
Along可以多玩一款航海类游戏,或者根据多款慢速页游的经验,思考一下大航海家3里面的哪些模式是好的,哪些模式会有问题,再者,思考一下这个游戏做成页游的情况下,是慢速(也就是一天只能做那么多事情),还是快速(只要你闲着点点点,总有事情做,有时间就会比没时间的人发展的好)比较好。

猩猩和狒狒,想想其他航海类或者其他你们玩过的游戏,有什么玩点可以加入。

猩猩考虑一下,如果我们打算这么发,Web,Windows(提示更新),Mac(提示更新),Android(提示更新apk),iOS(不动更,完全单独发包),有什么需要注意,有哪些地方会出问题。(比如web不能用xml,这就是个问题)。

狒狒,去看看我们各个地方的资源,哪些有好的来源,要从什么游戏里扒,我会给你个大航海时代4的拆包工具,可以把大航海4的资源拆出来,看看有没有可以利用的地方,界面怎么设计,才会又美观,又不会没有资源可做。

\subsubsection{十月底前(2015.10.08-2015.10.31)}
\paragraph{主要任务}
\subparagraph{全部}
整体流程完全跑通,协议通讯不再变动。
\subparagraph{数据库端}
彻底完成登录开发。
\subparagraph{客户端}
连通网络层,完成登录开发。

\subsubsection{十一月底前(2015.11.01-2015.11.30)}

\subsubsection{初始阶段收尾(2015.12.01-2015.12.18)}

\paragraph{}

\subparagraph{}

\section{第二阶段(2015.12.18-2016.01.08)}
\subsection{服务器加速阶段}

\subsubsection{寒假期间(2015.12.18-2016.01.08)}


\paragraph{}

\subparagraph{}

\end{document}